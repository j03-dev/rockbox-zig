% $Id$ %
\chapter{\label{ref:rockbox_interface}Quick Start}
\section{Basic Overview}
\subsection{The \daps{} controls}

% include the front image. Using \specimg makes this fairly easy,
% but requires to use the exact value of \specimg in the filename!
% The extension is selected in the preamble, so no further \Ifpdfoutput
% is necessary.
%
% The check looks for a png file -- we use png for the HTML manual, so that
% format needs to be present. It can also be used for the pdf manual, but
% usually we provide a pdf version of the file for that. Picking the correct
% one is done by LaTeX automatically, but for checking the filename we need to
% specify the extension.
\begin{center}
\IfFileExists{rockbox_interface/images/\specimg-front.png}
   {\Ifpdfoutput{\includegraphics[height=8cm,width=10cm,keepaspectratio=true]%
    {rockbox_interface/images/\specimg-front}}
   {\includegraphics{rockbox_interface/images/\specimg-front}}
   }
   {\color{red}{\textbf{WARNING!} Image not found}%
    \typeout{Warning: missing front image}
   }
\end{center}
\opt{HAVEREMOTEKEYMAP}{
  % spacing between the two pictures, could possibly be improved
  \begin{center}
  \IfFileExists{rockbox_interface/images/\specimg-remote.png}
    {\Ifpdfoutput{\includegraphics[height=5.6cm,width=10cm,keepaspectratio=true]{rockbox_interface/images/\specimg-remote}}
    {\includegraphics{rockbox_interface/images/\specimg-remote}}
    }
    {\color{red}{\textbf{WARNING!} Image not found}%
     \typeout{Warning: missing remote image}
    }
  \end{center}
}

Throughout this manual, the buttons on the \dap{} are labelled according to the
picture above.

\opt{erosqnative}{
  \note{The \dap{} has two outputs: Headphone output on the left and Line output on the right.
  Due to the circuitry which does detection, it is recommended to only use headphones plugged
  into the Headphone output and devices which take Line input into the Line output. If devices
  which take a Line input are plugged into the Headphone output, it is likely that the \dap{}
  will not detect the plug and no sound will result.

  In the other direction, headphones plugged into the Line output may damage the headphones (and
  your ears!) due to the extremely loud volume.

  Note that the volume of the Line output is set by the \setting{Volume Limit} if it needs to
  be reduced.}
}

\opt{touchscreen}{
The areas of the touchscreen in the 3$\times$3 grid mode are in turn referred as follows:
\begin{table}
    \centering
    \begin{tabular}{|c|c|c|}
	\hline
        \TouchTopLeft & \TouchTopMiddle & \TouchTopRight \\ [5ex]
	\hline
	\TouchMidLeft & \TouchCenter & \TouchMidRight \\ [5ex]
	\hline
	\TouchBottomLeft & \TouchBottomMiddle & \TouchBottomRight \\ [5ex]
	\hline
    \end{tabular}
\end{table}
}%
Whenever a button name is prefixed by ``Long'', a long press of approximately
one second should be performed on that button. The buttons are described in
detail in the following paragraph.
\blind{%
  Additional information for blind users is available on the Rockbox website at
  \wikilink{BlindFAQ}.

  %
  \opt{iriverh100}{
  Hold or lay the \dap{} so that the side with the joystick and LCD is facing
  towards you, and the curved side is at the top. The joystick functions as
  the \ButtonUp{}, \ButtonRight{}, \ButtonLeft{}, and \ButtonDown{} buttons when
  pressed in the appropriate direction. Pressing the joystick down functions as
  \ButtonSelect{}.
  On the right side of the \dap{} are the \ButtonOn{}, \ButtonOff{},
  \ButtonMode{} buttons, and the \ButtonHold{} switch. When this switch is
  switched towards the bottom of the \dap{}, hold is on, and none of the other
  buttons have any effect.

  On the left side is the \ButtonRec{} button. Above that is the internal microphone.

  On the top panel of the \dap{}, from left to right, you can find the
  following: headphone mini jack plug, remote port, Optical line-in, Optical line-out.

  On the bottom panel of the \dap{}, from left to right, you can find the
  following: power jack, reset switch, and USB port. In the event that your
  \dap{} hard locks, you can reset it by inserting a paper clip into the hole
  where the reset switch is.}
  %
  \opt{iriverh300}{
  Hold or lay the \dap{} so that the side with the button pad and
  LCD is facing towards you.  The buttons on the button pad are as follows:  top
  left corner: \ButtonOn{}, bottom left corner: \ButtonOff{}, top right corner:
  \ButtonRec, bottom right corner: \ButtonMode{}.  In the center of the button pad
  is a button labelled \ButtonSelect{}.  Surrounding the \ButtonSelect{} button are
  the \ButtonUp{}, \ButtonDown{}, \ButtonLeft{}, and \ButtonRight{} buttons.

  On the top panel of the \dap{}, from left to right, you can find the
  following: headphone mini jack plug, remote port, line-in, line-out.

  On the left hand side of the \dap{} is the internal microphone. Just underneath
  this is a small hole, the reset switch. In the event that your \dap{} hard locks,
  you can reset it by inserting a paper clip into the hole where the reset switch
  is.

  On the right hand side of the \dap{} is the \ButtonHold{} switch. When this is
  switched towards the bottom of the \dap{}, hold is on, and none of the other
  buttons have any effect.

  On the bottom panel of the \dap{}, from left to right, you can find the
  following:  power jack and two USB ports.  The USB port on the right is used
  to connect your \dap{} to your computer.  The USB port on the left is not
  used in Rockbox.
  }
  %
  \opt{mpiohd200}{
  Hold or lay the \dap{} so that the side with the LCD is facing towards you.
  On the right hand side there is a rocker switch at the top which serves as
  \ButtonRew{} and \ButtonFF{} when rocked up or down, respectively.
  Pressing the rocker in functions as the \ButtonFunc{} button. Below the rocker
  there are the \ButtonRec{} and \ButtonPlay{} buttons. At the bottom of the
  right panel there is the \ButtonHold{} switch. When this is switched towards the
  bottom of the \dap{}. hold is on, and none of the other buttons have any effect.

  On the top panel of the \dap{} there is another rocker which serves as the
  \ButtonVolDown{} and \ButtonVolUp{} buttons when pressed to the left or right,
  respectively.

  On the left hand side of the \dap{} there is a headphone mini jack plug at the top
  and a small hole at the bottom, the reset switch. In the event that your \dap{}
  hard locks, you can reset it by inserting a paper clip into the hole where the
  reset switch is.

  On the bottom panel of the \dap{}, from left to right, you can find the
  following: power jack, line-in jack and USB port (under rubber cover).
  }
  %
  \opt{ipod4g,ipodcolor,ipodvideo,ipodmini}{
  The main controls on the \dap{} are a slightly indented scroll wheel
  with a flat round button in the center. Hold the \dap{} with these controls
  facing you.

  The top of the player will have the following, from left to
  right:
  \opt{ipod4g,ipodcolor}{remote connector, headphone socket, \ButtonHold{}
    switch.}
  \opt{ipodvideo}{\ButtonHold{} switch, headphone socket.}
  \opt{ipodmini}{\ButtonHold{} switch, remote connector, headphone socket.}

  The dock connector that is used to connect your \dap{} to your computer is on
  the bottom panel of the \dap{}.

  The button in the middle of the wheel is called \ButtonSelect{}. You can
  operate the wheel by pressing the top, bottom, left or right sections,
  or by sliding your finger around it.  The top is \ButtonMenu{}, the bottom is
  \ButtonPlay{}, the left is \ButtonLeft{}, and the right is \ButtonRight{}.
  When the manual says to \ButtonScrollFwd{}, it means to slide your finger
  clockwise around the wheel. \ButtonScrollBack{} means to slide your finger
  counterclockwise. Note that the wheel is sensitive, so you will need to move
  slowly at first and get a feel for how it works.

  Note that when the \ButtonHold{} switch is pushed toward the center of the \dap{},
  hold is on, and none of the other controls do anything.  Be sure
  \ButtonHold{} is off before trying to use your player.
  }
  %
  \opt{ipod3g}{
  The main controls on the \dap{} are a slightly indented touch wheel
  with a flat round button in the center, and four buttons in a row above the
  touch wheel. Hold the \dap{} with these controls
  facing you.

  The top of the player will have the following, from left to
  right: remote connector, headphone socket, \ButtonHold{} switch.

  The dock connector that is used to connect your \dap{} to your computer is on
  the bottom panel of the \dap{}.

  The button in the middle of the wheel is called \ButtonSelect{}. You can
  operate the wheel by sliding your finger around it.  The row of
  buttons consists of, from left to right, the \ButtonLeft{},
  \ButtonMenu{}, \ButtonPlay{}, and \ButtonRight{} buttons.
  When the manual says to \ButtonScrollFwd{}, it means to slide your finger
  clockwise around the wheel. \ButtonScrollBack{} means to slide your finger
  counterclockwise. Note that the wheel is sensitive, so you will need to move
  slowly at first and get a feel for how it works.

  Note that when the \ButtonHold{} switch is pushed toward the center of the \dap{},
  hold is on, and none of the other controls do anything.  Be sure
  \ButtonHold{} is off before trying to use your player.
  }
  %
  \opt{ipod1g2g}{
  The main controls on the \dap{} are a slightly indented wheel
  with a flat round button in the center, and four buttons surrounding
  it. On the 1st generation iPod, this wheel physically turns. On the
  2nd generation iPod, this wheel is touch-sensitive. Hold the \dap{} with these controls
  facing you.

  The top of the player will have the following, from left to
  right: FireWire port, headphone socket, \ButtonHold{} switch.

  The FireWire port is used to connect your \dap{} to the computer and
  to charge its battery via a wall charger.

  The button in the middle of the wheel is called \ButtonSelect{}. You can
  operate the wheel by turning it, or sliding your finger around
  it. The top is \ButtonMenu{}, the bottom is \ButtonPlay{}, the left
  is \ButtonLeft{}, and the right is \ButtonRight{}.
  When the manual says to \ButtonScrollFwd{}, it means to slide your finger
  clockwise around the wheel. \ButtonScrollBack{} means to slide your finger
  counterclockwise. Note that the wheel is sensitive, so you will need to move
  slowly at first and get a feel for how it works.

  Note that when the \ButtonHold{} switch is pushed toward the center of the \dap{},
  hold is on, and none of the other controls do anything.  Be sure
  \ButtonHold{} is off before trying to use your player.
  }
  %
  \opt{ipodnano,ipodnano2g}{
  The main controls on the \dap{} are a slightly indented wheel with a
  flat round button in the center. Hold the \dap{} with these controls on the
  top surface. There is a \ButtonHold{} switch at one end, and
  headphone and dock connector at the other; be sure the end with the
  switch is facing away from you.

  The button in the middle of the wheel is called \ButtonSelect{}. You can
  operate the wheel by pressing the top, bottom, left or right sections,
  or by sliding your finger around it.  The top is \ButtonMenu{}, the bottom is
  \ButtonPlay{}, the left is \ButtonLeft{}, and the right is \ButtonRight{}.
  When the manual says to \ButtonScrollFwd{}, it means to slide your finger
  clockwise around the wheel. \ButtonScrollBack{} means to slide your finger
  counterclockwise. Note that the wheel is sensitive, so you will need to move
  slowly at first and get a feel for how it works.

  Note that when the \ButtonHold{} switch is pushed toward the center of the \dap{},
  hold is on, and none of the other controls do anything; be sure \ButtonHold{} is
  off before trying to use your player.
  }
  %
  \opt{iriverh10,iriverh10_5gb}{
  Hold or lay the \dap{} so that the side with the scroll pad and
  LCD is facing towards you. In the centre below the lcd is the scroll pad. It
  is oriented vertically. Touching the top and bottom half of it acts as the
  \ButtonScrollUp{}  and \ButtonScrollDown{} buttons respectively. On the left
  of the scroll pad is the \ButtonLeft{} button and on the right is the
  \ButtonRight{} button.

  There are three buttons on the right hand side of the \dap{}. From top to
  bottom, they are: \ButtonRew{}, \ButtonPlay{} and \ButtonFF{}. On the left
  hand side is the \ButtonPower{} button.

  On the top panel of the \dap{}, from left to right, you can find the
  following: \ButtonHold{} switch, \opt{iriverh10}{reset pin hole, }remote port
  and headphone mini jack plug.

  On the bottom panel of the \dap{} is the data cable port.}
  %
  \opt{gigabeatf}{
  \note{The following description is for the Gigabeat F, but can also apply for the
  Gigabeat X. The Gigabeat F is slightly larger and more rectangular shaped, while the
  Gigabeat X is smaller and has a slightly tapered back.}

  Hold the \dap{} with the screen on top and the controls on the right hand side.
  Below the screen is a cross-shaped touch sensitive pad which contains the
  \ButtonUp{}, \ButtonDown{}, \ButtonLeft{} and \ButtonRight{} controls.  On the
  Gigabeat X, this pad will feel slightly raised up, while it will feel slightly
  sunken in on the Gigabeat F. On the top of the unit, from left to right, are the
  power socket, the \ButtonHold{} switch, and the headphone socket.  The
  \ButtonHold{} switch puts the \dap{} into hold mode when it is switched to the
  right of the unit. The buttons will have no effect when this is the case.

  Starting from the left hand side on the bottom of the unit, nearer to the front
  than the back, is a recessed switch which
  controls whether the battery is on or off.  When this switch is to the left,
  the battery is disconnected.  This can be used for a hard reset of the unit,
  or if the \dap{} is being placed in storage.  Next to that is a connector for
  the docking station and finally on the right hand side of the bottom of the
  unit is a mini USB socket for connecting directly to USB.

  Finally on the right hand side of the unit are some control buttons.  Going from
  the bottom of the unit to the top there is a small round \ButtonA{} buttton then a
  rocker volume switch with of the \ButtonVolDown{} button below the \ButtonVolUp{}
  button.  Above that is are two more small round buttons, the \ButtonMenu{}
  button and nearest to the top of the unit the \ButtonPower{} button, which is held
  down to turn the \dap{} on or off. If you have a Gigabeat X, these buttons are small
  metallic buttons that are place further up on the right hand side, and closer
  together. The layout is still the same, however.}
  %
  \opt{gigabeats}{
  Hold the \dap{} with the screen on top and the controls on the right hand side.
  Directly below the bottom edge of the screen are two buttons, \ButtonBack{}
  on the left and \ButtonMenu{} on the right. Below them is a cross-shaped pad
  which contains the \ButtonUp{}, \ButtonDown{}, \ButtonLeft{}, \ButtonRight{}
  and \ButtonSelect{} controls.
  On the top of the unit from left to right are the headphone socket and the
  \ButtonHold{} switch.  The \ButtonHold{} switch puts the \dap{} into
  hold mode when it is switched to the right of the unit.
  The buttons will have no effect when this is the case.

  Starting from the left hand side on the bottom of the unit, nearer to the back
  than the front, is a recessed switch which controls whether the battery is on
  or off.  When this switch is to the left, the battery is disconnected.
  This can be used for a hard reset of the unit, or if the \dap{} is being placed
  in storage.  Next to that is a mini USB socket for connecting directly to USB,
  and finally a custom connector, presumably for planned accessories which were
  never released.

  Finally on the right hand side of the unit are some control buttons and the power
  connector.  Going from the bottom of the unit to the top, there is the power
  connector socket, followed by three small round buttons, the
  \ButtonNext{} buttton, \ButtonPlay{} button, and \ButtonPrev{} button (from bottom
  to top) then a rocker volume switch with of the \ButtonVolDown{} button below the
  \ButtonVolUp{} button.  Above that is one more small round button, the \ButtonPower{}
  button, which is held down to turn the \dap{} on or off.}
  %
  \opt{mrobe100}{
  Hold the \dap{} with the black front facing you such that the m:robe writing
  is readable. Below the writing is the touch sensitive pad with the
  \ButtonMenu{}, \ButtonPlay{}, \ButtonLeft{}, \ButtonRight{} and \ButtonDisplay
  controls indicated by their symbols. The dotted center strip is devided in
  three parts: \ButtonUp{}, \ButtonSelect{} and \ButtonDown. On the top of the
  unit, on the right, is the \ButtonPower{} switch, which is held down to turn
  the \dap{} on or off.

  The \ButtonHold{} switch is located on the left of the \dap{}, below the
  headphone socket. It puts the \dap{} into hold mode when it is switched to the
  top of the unit. The buttons will have no effect when this is the case. On the
  bottom of the unit, there is a connector for the docking station or the
  proprietary USB connector for connecting directly to USB.}
  %
  \opt{iaudiom5,iaudiox5}{
  The \dap{} is curved so that the end with the screen on it is thicker than the
  other end.  Hold the \dap{} wih the thick end towards the top and the screen
  facing towards you.  Half way up the front of the unit on the right hand side
  is a four way joystick which is the \ButtonUp{}, \ButtonDown{},
  \ButtonLeft{}, and \ButtonRight{} buttons. When pressed it serves as \ButtonSelect{}.

  On the right hand side of the \dap{} from top to bottom, first there is a two
  way switch.  the \ButtonPower{} button is activated by pushing this switch up,
  and pushing this switch down until it clicks slightly will activate the
  \ButtonHold{} button.  When the switch is in this position, none of the other
  keys will have an effect.

  Below the switch is a lozenge shaped button which is the \ButtonRec{}
  button, and below that the final button on this side of the unit, the
  \ButtonPlay{} button.  Just below this is a small hole which is difficult to
  locate by touch which is the internal microphone.  At the very bottom of
  this side of the unit is the reset hole, which can be used to perform a hard
  reset by inserting a paper clip.

  On the bottom of the unit is the connector for the
  \playerman{} subpack or dock.  On the top of the unit is a charge
  indicator light, which may feel a bit like a button, but is not.

  From the top of the \dap{} on the left hand side is the headphone socket, then the
  remote connector.  Below this is a cover which protects the \opt{iaudiox5}{USB
  host connector.}\opt{iaudiom5}{USB and charging connector}.}
  %
  \opt{e200,e200v2}{
  Hold the \dap{} with the turning wheel at the front and bottom.  On the bottom left
  of the front of the \dap{} is a raised round button, the \ButtonPower{} button.
  Above and to the left of this, on the outside of the turning wheel are four
  buttons.  These are the \ButtonUp{}, \ButtonDown{}, \ButtonLeft{} and
  \ButtonRight{} buttons.  Inside the wheel is the \ButtonSelect{} button.  Turning
  the wheel to the right activates the \ButtonScrollFwd{} function, and to the
  left, the \ButtonScrollBack{} function.

  On the right of the unit is a slot for inserting flash cards.  On the bottom is
  the connector for the USB cable.  On the left is the \ButtonRec{} button, and
  on the top, there is the headphone socket to the right, and the \ButtonHold{}
  switch.  Moving this switch to the right activates hold mode in which none of the
  other buttons have any effect.  Just to the left of the \ButtonHold{} switch is a
  small hole which contains the internal microphone.}
  %
  \opt{c200,c200v2}{
  Hold the \dap{} with the buttons on the right and the screen on the left. On
  the right side of the unit, there is a series of four connected buttons that
  form a square. The four sides of the square are the \ButtonUp{},
  \ButtonDown{}, \ButtonLeft{} and \ButtonRight{} buttons, respectively. Inside
  the square formed by these four buttons is the \ButtonSelect{} button. At the
  bottom right corner of the square is a small separate button, the
  \ButtonPower{} button.

  Moving clockwise around the outside of the unit, on the top are the \ButtonVolUp{}
  and \ButtonVolDown{} buttons, which control the volume of playback. The buttons can
  be distinguished by a sunken triangle on the \ButtonVolDown{} button, and a
  raised triangle on the \ButtonVolUp{} button. To the right of
  the volume buttons on the top of the unit is the slot for inserting flash
  memory cards. On the right side of the unit is the connector for the USB
  cable. At center of the bottom of the \dap{} is the \ButtonRec{} button. To
  the left of the \ButtonRec{} button is the \ButtonHold{} switch. Moving this
  switch to the right activates hold mode, in which none of the other buttons
  have any effect. On the lower left side of the unit is the headphone socket.
  Immediately above the headphone socket is a lanyard loop and the microphone.
  }
  %
  \opt{fuze,fuzev2}{
  Hold the \dap{} with the controls on the bottom and the screen on the top. The main
  controls are a scroll wheel with four clickable points and a button in the centre; pressing
  this centre button functions as \ButtonSelect{}. Going clockwise from the top, the clickable
  points on the wheel are the \ButtonUp{}, \ButtonRight{}, \ButtonDown{}, and \ButtonLeft{}
  buttons. Turning the wheel clockwise is \ButtonScrollFwd{}, and turning it counter-clockwise
  is \ButtonScrollBack{}. Immediately above and to the right of the wheel is the \ButtonHome{}
  button.

  On the lower left of the unit is a slot for inserting microSD cards. Immediately below that is
  the opening for the microphone.

  On the bottom of the unit is the connector for connecting a USB cable and the headphone socket.
  On the lower right hand side of the unit is a two-way switch. Pressing this switch up acts as
  \ButtonPower{}, and clicking it down until it locks acts as the \ButtonHold{} switch. When the
  \ButtonHold{} switch is on, none of the other buttons have any effect.
  }
  %
  \opt{clipplus,clipv1,clipv2,clipzip}{
  Hold the \dap{} with the controls on the bottom and the screen on the top. The main
  controls are a four-way pad with a button in the centre; pressing this centre button
  functions as \ButtonSelect{}. Going clockwise from the top, the four-way pad contains
  the \ButtonUp{}, \ButtonRight{}, \ButtonDown{}, and \ButtonLeft{} buttons.
  Immediately above and to the \nopt{clipzip}{right}\opt{clipzip}{left} of the four-way
  pad is the \ButtonHome{} button.
  }
  %
  \opt{clipplus,clipzip}{
  The \ButtonPower{} button is on the top of the \dap{}\opt{clipplus}{, towards the right side.}

  At the bottom of the right side of the \dap{} is a slot for microSD cards.
  Above this slot on the right side is the headphone socket.

  On the left hand panel is a two-way button that acts as \ButtonVolDown{} when
  pressed on the bottom, and \ButtonVolUp{} when pressed on the top. Immediately
  above the switch is a mini-USB port to connect the \dap{} to a computer.

  }
  %
  \opt{clipv1,clipv2}{
  On the left hand panel is a two way switch. Pressing this switch up acts as
  \ButtonPower{}, and clicking it down until it locks acts as the \ButtonHold{}
  switch. When the \ButtonHold{} switch is on, none of the other buttons have any
  effect. Immediately above the switch is a mini-USB port to connect the \dap{} to
  a computer.

  On the right hand panel is a two-way button that acts as \ButtonVolDown{} when
  pressed on the bottom, and \ButtonVolUp{} when pressed on the top. Immediately
  above this button is the headphone socket.
  }
  %
  \opt{vibe500}{
  Hold or lay the \dap{} so that the side with the controls and
  LCD is facing towards you. Below the LCD is the touch sensitive pad with the \ButtonMenu{},
  \ButtonPlay{}, \ButtonLeft{}, \ButtonRight{} controls and the scroll pad in the centre. The
  scroll pad is oriented vertically between the \ButtonOK{} and \ButtonCancel{} buttons.
  Sliding a finger up or down the scroll pad acts as \ButtonUp{} and \ButtonDown{} respectively.
  Note that the scroll pad is sensitive, so you will need to move
  slowly at first and get a feel for how it works.

  There are two buttons on the right hand side of the \dap{}: \ButtonPower{} on the top and
  \ButtonRec{} underneath. Under these buttons, from top to bottom you can find: USB connector,
  power connector and the reset hole if you need to perform a hardware reset.

  The \ButtonHold{} switch is located on the left hand side of the \dap{}. Note that when the
  \ButtonHold{} switch is moved towards the top of the \dap{}, hold is turned on and all the
  other controls are disabled. Be sure \ButtonHold{} is off before trying to use your player.

  On the top on the \dap{} is the internal microphone on the left and the line-in socket on the
  right, near the headphone socket.}
  %
  \opt{samsungyh820}{
  Hold or lay the \dap{} so that the side with the controls and
  LCD is facing towards you. Directly below the bottom edge of the screen are three buttons:
  \ButtonRew{} on the left, \ButtonPlay{} in the middle and \ButtonFF{} on the right. Below them
  is a four-way pad which contains the \ButtonDown{}, \ButtonUp{}, \ButtonLeft{} and
  \ButtonRight{} controls.

  At the top of the right hand side of the \dap{} is the \ButtonRec{} button.

  On the top panel of the \dap{}, from left to right, you can find the following: headphone
  socket, line-in socket, internal microphone, and the \ButtonHold{} switch. Note that when the
  \ButtonHold{} switch is moved towards the center of the \dap{}, hold is turned on and all the
  other controls are disabled. Be sure \ButtonHold{} is off before trying to use your player.

  At the top of the back side of the player, just under the \ButtonHold{} button is the reset
  hole, if you need to perform a hardware reset.

  The USB/dock connector that is used to connect your \dap{} to your computer is on
  the bottom panel of the \dap{}.
  }
  %
  \opt{samsungyh920,samsungyh925}{
  Hold or lay the \dap{} so that the side with the controls and
  LCD is facing towards you. Below the LCD is a four-way pad with the \ButtonDown{},
  \ButtonUp{}, \ButtonLeft{} and \ButtonRight{} buttons.

  There are three buttons at the top of the right hand side of the \dap{}: \ButtonFF{} on the top,
  \ButtonPlay{} in the middle and \ButtonRew{} underneath. Below these buttons is the \ButtonRec{}
  switch. Rockbox doesn't take note of the actual \emph{position} of the switch, but reacts to a
  \emph{switching movement} like pressing a regular button.

  On the top panel of the \dap{}, from left to right, you can find the following: headphone/remote
  socket, line-in socket, internal microphone, and the \ButtonHold{} switch. Note that when the
  \ButtonHold{} switch is moved towards the center of the \dap{}, hold is turned on and all the
  other controls are disabled. Be sure \ButtonHold{} is off before trying to use your player.

  At the top of the back side of the player, just under the \ButtonHold{} button is the reset hole,
  if you need to perform a hardware reset.

  The USB/dock connector that is used to connect your \dap{} to your computer is on
  the bottom panel of the \dap{}.
  }
  %
}

\subsection{Turning the \dap{} on and off}
\opt{cowond2}{Rockbox has a dual-boot feature with the original firmware being
  the default.\\}
To turn on and off your Rockbox enabled \dap{} use the following keys:
    \begin{btnmap}
      \opt{IRIVER_H100_PAD,IRIVER_H300_PAD}{\ButtonOn}%
      \opt{MPIO_HD200_PAD,MPIO_HD300_PAD,SAMSUNG_YH92X_PAD,SAMSUNG_YH820_PAD}%
          {Long \ButtonPlay}%
      \opt{IPOD_4G_PAD}{\ButtonMenu{} / \ButtonSelect}%
      \opt{IPOD_3G_PAD}{\ButtonMenu{} / \ButtonPlay}%
      \opt{IAUDIO_X5_PAD,IRIVER_H10_PAD,SANSA_E200_PAD,SANSA_C200_PAD,ONDA_VX777_PAD%
          ,GIGABEAT_PAD,MROBE100_PAD,GIGABEAT_S_PAD,sansaAMS,PBELL_VIBE500_PAD%
          ,SANSA_FUZEPLUS_PAD,XDUOO_X3_PAD,AIGO_EROSQ_PAD%
          }{\ButtonPower}%
      \opt{COWON_D2_PAD} {\ButtonPower{}, then \ButtonHold}%
      \opt{ONDA_VX777_PAD} {\ButtonPower{}}%
      \opt{AGPTEK_ROCKER_PAD}{\ButtonPower{}}%
          &
      \opt{HAVEREMOTEKEYMAP}{
          \opt{IRIVER_RC_H100_PAD}{\ButtonRCOn}%
          \opt{IAUDIO_RC_PAD}{\ButtonRCPlay}
          &}

      Start Rockbox
          \\

      \opt{IRIVER_H100_PAD,IRIVER_H300_PAD}{Long \ButtonOff}%
      \opt{MPIO_HD200_PAD,MPIO_HD300_PAD,SAMSUNG_YH92X_PAD,SAMSUNG_YH820_PAD}%
          {Long \ButtonPlay}%
      \opt{IPOD_4G_PAD,IPOD_3G_PAD}{Long \ButtonPlay}%
      \opt{IAUDIO_X5_PAD,IRIVER_H10_PAD,SANSA_E200_PAD,SANSA_C200_PAD%
          ,GIGABEAT_PAD,MROBE100_PAD,GIGABEAT_S_PAD,sansaAMS,COWON_D2_PAD%
          ,PBELL_VIBE500_PAD,ONDA_VX777_PAD,SANSA_FUZEPLUS_PAD,XDUOO_X3_PAD,AIGO_EROSQ_PAD%
          }{Long \ButtonPower}%
      \opt{AGPTEK_ROCKER_PAD}{Long \ButtonPower{}}%
          &
      \opt{HAVEREMOTEKEYMAP}{
          \opt{IRIVER_RC_H100_PAD}{Long \ButtonRCStop}%
          \opt{IAUDIO_RC_PAD}{Long \ButtonRCPlay}
          &}

      Shutdown Rockbox
          \\
    \end{btnmap}

\label{ref:Safeshutdown}On shutdown, Rockbox automatically saves its settings.

\opt{IRIVER_H100_PAD,IRIVER_H300_PAD,IAUDIO_X5_PAD,SANSA_E200_PAD%
  ,SANSA_C200_PAD,IRIVER_H10_PAD,IPOD_4G_PAD,GIGABEAT_PAD}{%
  If you have problems with your settings, such as accidentally having
  set the colours to black on black, they can be reset at boot time.  See
  the Reset Settings in \reference{ref:manage_settings_menu} for details.
}%

\opt{GIGABEAT_PAD,IPOD_4G_PAD,SANSA_E200_PAD%
,SANSA_C200_PAD,IAUDIO_X5_PAD,IAUDIO_M5_PAD,IPOD_3G_PAD}{%
  In the unlikely event of a software failure, hardware poweroff or reset can be
  performed by holding down
  \opt{GIGABEAT_PAD}{the battery switch}\opt{IPOD_4G_PAD}
  {\ButtonMenu{} and \ButtonSelect{} simultaneously}%
  \opt{IPOD_3G_PAD}{\ButtonMenu{} and \ButtonPlay{} simultaneously}%
  \opt{SANSA_E200_PAD,SANSA_C200_PAD,IAUDIO_X5_PAD,IAUDIO_M5_PAD}
  {\ButtonPower} until the \dap{} shuts off or reboots.
}%
\opt{IRIVER_H100_PAD,IRIVER_H300_PAD,IAUDIO_M3_PAD,IRIVER_H10_PAD,MROBE100_PAD
    ,PBELL_VIBE500_PAD,MPIO_HD200_PAD,MPIO_HD300_PAD,SAMSUNG_YH92X_PAD%
    ,SAMSUNG_YH820_PAD,XDUOO_X3_PAD}{%
  In the unlikely event of a software failure, a hardware reset can be
  performed by inserting a paperclip gently into the Reset hole.
}%

\nopt{gigabeatf,iaudiom3,iaudiom5,iaudiox5}
  {
  \subsection{Starting the original firmware}
  \label{ref:Dualboot}
  \opt{ipod4g,ipodcolor,ipodvideo,ipodnano,ipodnano2g,ipodmini}
    {
    Rockbox has a dual-boot feature. To boot into the original firmware, shut
    down the device as described above. Turn on the \ButtonHold{} switch
    immediately after turning the player on. The Apple logo will
    display for a few seconds as Rockbox loads the original firmware.

    You can also load the original firmware by shutting down the device,
    then clicking the \ButtonHold{} switch on and connecting the iPod
    to your computer.

    Regardless of which method you use to boot to the original firmware, you can
    return to Rockbox by pressing and holding \ButtonMenu{} and \ButtonSelect{}
    simultaneously until the player hard resets.
    }

  \opt{ipod1g2g,ipod3g}
    {
    Rockbox has a dual-boot feature. To boot into the original firmware, shut
    down the device as described above. Turn on the \ButtonHold{} switch
    immediately after turning the player on. The Apple logo will
    display for a few seconds as Rockbox loads the original firmware.

    You can also load the original firmware by shutting down the device,
    then clicking the \ButtonHold{} switch on and connecting the iPod
    to your computer.

    Regardless of which method you use to boot to the original firmware, you can
    return to Rockbox by pressing and holding \ButtonMenu{} and \ButtonPlay{}
    simultaneously until the player hard resets.
    }

  \opt{iriverh100,iriverh300}
    {
    Rockbox has a dual-boot feature. To boot into the original firmware,
    when the \dap{} is turned off, press and hold the \ButtonRec{} button,
    and then press the \ButtonOn{} button.
    }
  \opt{fuzeplus}
    {
    Rockbox has a dual-boot feature. To boot into the original firmware,
    when the \dap{} is turned off, press and hold the \ButtonVolDown{} button,
    and then press and hold the \ButtonPower{} button while keeping the
    \ButtonVolDown{} button pressed. After 5 to 10 seconds the original
    firmware should boot.

    It is also possible to connect your \dap{} to your computer using the
    original firmware. To do so you may press and hold the \ButtonVolDown{}
    button and connect your device to the computer while keeping the
    \ButtonVolDown{} button pressed. After 5 to 10 seconds the original
    firmware should boot into USB mode.
    }
  \opt{mpiohd200,mpiohd300}
    {
    Rockbox has a dual-boot feature. To boot into the original firmware,
    when the \dap{} is turned off, press and hold the \ButtonRec{} button,
    and then press the \ButtonPlay{} button. This will bring you to the
    short menu where you can choose among: Boot Rockbox, Boot MPIO firmware
    and Shutdown. Select the option you need with \ButtonRew{} and \ButtonFF{}
    and confirm with long \ButtonPlay{}.
    }
  \opt{iriverh10,iriverh10_5gb}
    {
    Rockbox has a dual-boot feature. It loads the original firmware from
    the file \fname{/System/OF.mi4}. To boot into the original firmware,
    press and hold the \ButtonLeft{} button while turning on the player.
    \note{The iriver firmware does not shut down properly when you turn it off,
    it only goes to sleep. To get back into Rockbox when exiting from the
    iriver firmware, you will need to reset the player by \opt{iriverh10}{%
    inserting a pin in the reset hole}\opt{iriverh10_5gb}{removing and
    reinserting the battery}.}
    }

  \opt{sansa,sansaAMS}
    {
    Rockbox has a dual-boot feature. To boot into the original firmware,
    press and hold the \ButtonLeft{} button while turning on the player.
    }

  \opt{clipv2,fuzev2,clipplus}
    {
        \note{Rockbox does not boot into the original firmware when powered by
        a USB connection. Older versions of Rockbox do not provide USB support.
        If you have such a version installed you need to manually boot into the
        original firmware for data transfer via USB.}
    }

  \opt{mrobe100}
    {
    Rockbox has a dual-boot feature. It loads the original firmware from
    the file \fname{/System/OF.mi4}. To boot into the original firmware,
    when the \dap{} is turned off, press the \ButtonPower{} button once and then
    a second time when the m:robe bootlogo (the headphone) appears. Hold the
    \ButtonPower{} button until you see the ``Loading original firmware...''
    message on the screen.
    }

  \opt{gigabeats}
    {
    Rockbox has a dual-boot feature. To boot into the original firmware,
    turn the \ButtonHold{} switch on just after turning on the \dap{}.
    To return to Rockbox, shutdown the \dap{}, then turn the battery switch
    on the bottom off then on again. Rockbox should now start.
    }

  \opt{cowond2}
    {
    Use \ButtonPower{} to boot the original \playerman{} firmware.
    }

  \opt{vibe500}
    {
    Rockbox has a dual-boot feature where it is possible to load the original firmware from
    the file \fname{/System/OF.mi4}. To boot into the original firmware press and release
    \ButtonPower{} and then immediately after the backlight turns on, press the \ButtonOK{}
    button and keep it pressed until the original firmware starts.
    }

  \opt{samsungyh}
    {
    Rockbox has a dual-boot feature. It loads the original firmware from
    the file \fname{/System/OF.mi4}. To boot into the original firmware, press and hold
    for awhile the \ButtonPlay{} button and then immediately after the Samsung logo appears,
    press the \ButtonLeft{} button and keep it pressed until the original firmware starts.
    }

  \opt{ondavx777}
    {
    Rockbox has a dual-boot feature where it is possible to load the original firmware from
    the file \fname{/SD/ccpmp.bin}. To boot into the original firmware press and release
    \ButtonPower{} immediately after the Rockbox Logo appear on the screen.
    }

  \opt{xduoox3}
    {
    Rockbox has a dual-boot feature. To boot into the original firmware,
    when the \dap{} is turned off, set the \ButtonLock{} switch to locked,
    and then press the \ButtonPower{} button.
    }

  \opt{fiiom3k,shanlingq1,erosqnative}
    {
    Rockbox has a dual-boot feature. To boot into the original firmware,
    hold \ActionBootOFPlayer{} when powering on the \dap{}.

    \nopt{erosqnative}{
      You can trigger a normal \playerman{} firmware update by holding
      \ActionBootOFRecovery{} when powering on the \dap{}.
      \warn{Updating the original firmware will \textbf{erase} the Rockbox
      bootloader.}
    }

    \subsection{Entering the recovery menu}
    You can access the Rockbox bootloader's ``recovery menu'' by holding
    \ActionBootRecoveryMenu{}. This menu can be used to connect your \dap{}
    over USB to transfer files, update the Rockbox bootloader, or revert to a
    bootloader you've previously backed up.
    }

  }
\subsection{Putting music on your \dap{}}

\opt{usb_hid}{
\note{Due to a bug in some OS X versions, the \dap{} can not be mounted, unless
    the USB HID feature is disabled. See \reference{ref:USB_HID} for more
    information.\newline
}
}

With the \dap{} connected to the computer as an MSC/UMS device (like a
USB Drive), music files can be put on the player via any standard file
transfer method that you would use to copy files between drives (e.g. Drag-and-Drop).
Files may be placed wherever you like on the \dap{}, but it is strongly
suggested \emph{NOT} to put them in the \fname{/.rockbox} folder and instead
put them in any other folder, e.g. \fname{/}, \fname{/music} or \fname{/audio}.
The default directory structure that is assumed by some parts of Rockbox
\opt{albumart}{%
    (album art searching, and missing-tag fallback in some WPSes) uses the
    parent directory of a song as the Album name, and the parent directory of
    that folder as the Artist name. WPSes may display information incorrectly if
    your files are not properly tagged, and you have your music organized in a
    way different than they assume when attempting to guess the Artist and Album
    names from your filetree. See \reference{ref:album_art} for the requirements
    for Album Art to work properly.
}%
\nopt{albumart}{%
    (missing-tag fallback in some WPSes) uses the parent directory of a song
    as the Album name, and the parent directory of that folder as the Artist
    name. WPSes may display
    information incorrectly if your files are not properly tagged, and you have
    your music organized in a way different than they assume when attempting to
    guess the Artist and Album names from your filetree.
}%
    See \reference{ref:Supportedaudioformats} for a list of supported audio
    formats.

\subsection{The first contact}

After you have first started the \dap{}, you'll be presented by the
\setting{Main Menu}. From this menu you can reach every function of Rockbox,
for more information (see \reference{ref:main_menu}). To browse the files
on your \dap{}, select \setting{Files} (see \reference{ref:file_browser}), and to
browse in a view that is based on the meta-data\footnote{ID3 Tags, Vorbis
comments, etc.} of your audio files, select \setting{Database} (see
\reference{ref:database}).

\subsection{Basic controls}
When browsing files and moving through menus you usually get a list view
presented. The navigation in these lists are usually the same and should be
pretty intuitive.
In the tree view use \ActionStdNext{} and \ActionStdPrev{} to move around
the selection. Use \ActionStdOk{} to select an item. \opt{wheel_acceleration}{
Note that the scroll speed is accelerating the faster you rotate the wheel.}
When browsing the file system selecting an audio file plays it. The view
switches to the ``While playing screen'', usually abbreviated as ``WPS'' (see
\reference{ref:WPS}. The dynamic playlist gets replaced with the contents of
the current directory. This way you can easily treat directories as playlists.
The created dynamic playlist can be extended or modified while playing. This is
also known as ``on-the-fly playlist''.
To go back to the \setting{File Browser} stop the playback with the
\ActionWpsStop{} button or return to the file browser while keeping playback
running using \ActionWpsBrowse{}.
In list views you can go back one step with \ActionTreeParentDirectory.

\subsection{Basic concepts}
\subsubsection{Playlists}
Rockbox is playlist oriented. This means that every time you play an audio file,
a so-called ``dynamic playlist'' is generated, unless you play a saved
playlist. You can modify the dynamic playlist while playing and also save
it to a file. If you do not want to use playlists you can simply play your
files directory based.
Playlists are covered in detail in \reference{ref:working_with_playlists}.

\subsubsection{Menu}
From the menu you can customise Rockbox. Rockbox itself is very customisable.
Also there are some special menus for quick access to frequently used
functions.

\subsubsection{Context Menu}
Some views, especially the file browser and the WPS have a context menu.
From the file browser this can be accessed with \ActionStdContext{}.
The contents of the context menu vary, depending on the situation it gets
called. The context menu itself presents you with some operations you can
perform with the currently highlighted file. In the file browser this is
the file (or directory) that is highlighted by the cursor. From the WPS this is
the currently playing file. Also there are some actions that do not apply
to the current file but refer to the screen from which the context menu
gets called. One example is the playback menu, which can be called using
the context menu from within the WPS.

\section{Customising Rockbox}
Rockbox' User Interface can be customised using ``Themes''. Themes usually
only affect the visual appearance, but an advanced user can create a theme
that also changes various other settings like file view, LCD settings and
all other settings that can be modified using \fname{.cfg} files. This topic
is discussed in more detail in \reference{ref:manage_settings}.
The Rockbox distribution comes with some themes that should look nice on
your \dap{}.

\note{Some of the themes shipped with Rockbox need additional
fonts from the fonts package, so make sure you installed them.
Also, if you downloaded additional themes from the Internet make sure you
have the needed fonts installed as otherwise the theme may not display
properly.}

  \opt{usb_power}{
    \section{USB Charging}
    Your \dap{} will automatically charge when connected to USB. By default
    Rockbox will connect in mass storage mode to transfer files, but you can
    prevent this by holding down any button while plugging in the USB cable,
    or by changing the \setting{USB Mode} setting to \setting{Charge Only}.
    \nopt{fuzeplus}{
    \note{Be aware that holding a button may still perform its normal function,
    so it is recommended to use a button without harmful side effects, such as
    \ActionStdUsbCharge{}.}
    }
  }

\input{rockbox_interface/browsing_and_playing.tex}
