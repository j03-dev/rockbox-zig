\subsection{LastFm Scrobbler}
The \setting{LastFm Scrobbler} plugin enables you to parse the rockbox
playback log for tracks you have played for your own logging or upload
to scrobbling services, such as Last.fm, Libre.fm or ListenBrainz.

\setting{Playback Logging} must be enabled to record the tracks played
the plugin will ask you to enable logging if run with logging disabled.

\subsubsection{Menu}
\begin{itemize}
\item Remove duplicates - Only keeps the same track with the most time elapsed.
\item Delete playback log - Remove the current playback log once it has been read.
\item Save threshold - Percentage of track played to be considered 'L'istened.
\item Minimum elapsed (ms) - Tracks played less than this will not be recorded in log.
\item View log - View the current playback log.
\item Revert to Default - Default settings restored.
\item Cancel - Exit, you will be asked to save any changes
\item Export - Append scrobbler log and save any changes, not visible if no playback log exists.
\end{itemize}

After the plugin has exported the scrobbler log you can find it in the root
of the drive '.scrobbler.log' open it in the file browser to view the log.

Subsequent exports will be appended to .scrobbler.log thus Delete playback log is advised.

\begin{verbatim}
A copy of the log can be found in
'/rockbox/playback.old'
it will be overwritten with each export
\end{verbatim}

\subsubsection{Format}
Data will be saved in Audioscrobbler spec at: (use wayback machine).
\url{http://www.audioscrobbler.net/wiki/Portable_Player_Logging}.

\begin{verbatim}
The scrobbler format consists of the following data items
tab '\t' separated followed by newline '\n'
\end{verbatim}

\begin{itemize}
\item ARTIST
\item ALBUM
\item TITLE
\item TRACKNUM
\item LENGTH
\item RATING
\item TIMESTAMP
\item MUSICBRAINZ-TRACKID
\end{itemize}

If track info is not available (due to missing file or format limitations)
the track path will be used instead.
